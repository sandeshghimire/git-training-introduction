% sample.tex
\documentclass{beamer}
\usecolortheme[RGB={200,100,80}]{structure} 
\usetheme{Rochester}
\usepackage{pgf,tikz}
\usepackage{url}
\usepackage{listings}


\begin{document}


\title{Introduction to Git}
\author{Sandesh Ghimire\\
  San Jose, CA,\\
  {sandeshghimire@outlook.com}}
\date \today
\maketitle


%
% New Page 
%
%
%


%
% New Page 
%
%
%
\begin{frame}{what is Git}
\pause

Git is a tool for tracking changes made to a set of files over time, a task
traditionally known as �version control.� Although it is most often used by 
programmers to coordinate changes to software source code, and it is especially 
good at that, you can use Git to track any kind of content at all. Any body 
of related files evolving over time, which we�ll call a �project,� is a 
candidate for using Git
\end{frame}


%
% New Page 
%
%
%
\begin{frame}{Introduction to Git}
\pause
git add  \pause

git commit \pause

git init \pause 

git push   \pause

git pull \pause

git branch \pause 

git merge  \pause

git status \pause

git something \pause 


git add  \pause

git commit \pause

git init \pause 

\end{frame}



%
% New Page 
%
%
%
\begin{frame}{Introduction to Git}
\pause
git add  

\end{frame}

%
% New Page 
%
%
%
\begin{frame}{git commands}
Here is what \emph{itemized} and \emph{enumerated} lists look like:
\pause
\begin{columns}
  \begin{column}{0.45\textwidth}
  \begin{itemize}
    \item itemized item 1 \pause
    \item itemized item 2 \pause
    \item itemized item 3 \pause
  \end{itemize}
  \end{column}
  
  \begin{column}{0.45\textwidth}
  \begin{itemize}
    \item itemized item 1 \pause
    \item itemized item 2 \pause
    \item itemized item 3 \pause
  \end{itemize}
  \end{column}
\end{columns}
\end{frame}


%
% New Page 
%
%
%
\begin{frame}{Introduction to }
\pause
\begin{columns}
  \begin{column}{0.45\textwidth}
  \begin{itemize}
    \item itemized item 1 \pause
    \item itemized item 2 \pause
    \item itemized item 3 \pause
  \end{itemize}
  \end{column}
  
  \begin{column}{0.45\textwidth}
  \begin{itemize}
    \item itemized item 1 \pause
    \item itemized item 2 \pause
    \item itemized item 3 \pause
  \end{itemize}
  \end{column}
\end{columns}
\end{frame}


%
% New Page 
%
%
%
\begin{frame}{With Git, you can:}
\pause

\end{frame}

%
% New Page 
%
%
%

\begin{frame}{Introduction to Git}
\pause
\begin{itemize}
\item This one is always shown
\item<1-> The first time
\item<2-> The second time
\item<1-> Also the first time
\only<1-> This one is shown at the first time, but it will hide soon.
\end{itemize}
\end{frame}


%
% New Page 
%
%
%
\begin{frame}{Introduction to Git}
\pause
\end{frame}







%
% New Page 
%
%
%
\begin{frame}{Introduction to Git}
\pause
\end{frame}




%
% New Page 
%
%
%
\begin{frame}{Introduction to Git}
\pause
\end{frame}




%
% New Page 
%
%
%
\begin{frame}{Introduction to Git}
\pause
\end{frame}




%
% New Page 
%
%
%
\begin{frame}{Introduction to Git}
\pause
\end{frame}



%
% New Page 
%
%
%
\begin{frame}{Introduction to Git}
\pause
\end{frame}



%
% New Page 
%
%
%
\begin{frame}{Introduction to Git}
\pause
\end{frame}


%
% New Page 
%
%
%
\begin{frame}{Introduction to Git}
\pause
\end{frame}

%
% New Page 
%
%
%
\begin{frame}{Introduction to Git}
\pause
\end{frame}


%
% New Page 
%
%
%
\begin{frame}{Introduction to Git}
\pause
\end{frame}


%
% New Page 
%
%
%
\begin{frame}{Introduction to Git}
\pause
\end{frame}


%
% New Page 
%
%
%
\begin{frame}{Introduction to Git}
\pause
\end{frame}


%
% New Page 
%
%
%
\begin{frame}{Introduction to Git}
\pause
\end{frame}


%
% New Page 
%
%
%
\begin{frame}{Introduction to Git}
\pause
\end{frame}


%
% New Page 
%
%
%
\begin{frame}{Introduction to Git}
\pause
\end{frame}



%
% New Page 
%
%
%
\begin{frame}{Introduction to Git}
\pause
\end{frame}

%
% New Page 
%
%
%
\begin{frame}{Introduction to Git}
\pause
\end{frame}


%
% New Page 
%
%
%
\begin{frame}{Introduction to Git}
\pause
\end{frame}


%
% New Page 
%
%
%
\begin{frame}{Introduction to Git}
\pause
\end{frame}


%
% New Page 
%
%
%
\begin{frame}{Introduction to Git}
\pause
\end{frame}


%
% New Page 
%
%
%
\begin{frame}{Introduction to Git}
\pause
\end{frame}


%
% New Page 
%
%
%
\begin{frame}{Introduction to Git}
\pause
\end{frame}


%
% New Page 
%
%
%
\begin{frame}{Introduction to Git}
\pause
\end{frame}


%
% New Page 
%
%
%
\begin{frame}{Introduction to Git}
\pause
\end{frame}


%
% New Page 
%
%
%
\begin{frame}{Introduction to Git}
\pause
\end{frame}





\end{document}
